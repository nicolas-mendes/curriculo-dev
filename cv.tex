%----------------------------------------------------------------------------------------
%	DOCUMENT DEFINITION
%----------------------------------------------------------------------------------------
\documentclass[a4paper,12pt]{article}

%----------------------------------------------------------------------------------------
%	PACKAGES
%----------------------------------------------------------------------------------------
\usepackage{url}
\usepackage{parskip} 	

%other packages for formatting
\RequirePackage{color}
\RequirePackage{graphicx}
\usepackage[usenames,dvipsnames]{xcolor}
\usepackage[scale=0.9]{geometry}

%tabularx environment
\usepackage{tabularx}

%for lists within experience section
\usepackage{enumitem}

% centered version of 'X' col. type
\newcolumntype{C}{>{\centering\arraybackslash}X} 

%to prevent spillover of tabular into next pages
\usepackage{supertabular}
\newlength{\fullcollw}
\setlength{\fullcollw}{0.47\textwidth}

%custom \section
\usepackage{titlesec}				
\usepackage{multicol}
\usepackage{multirow}
\usepackage{booktabs}

\titleformat{\section}{\Large\scshape\raggedright}{}{0em}{}[\titlerule]
\titlespacing{\section}{0pt}{10pt}{10pt}

%for publications
\usepackage[style=authoryear,sorting=ynt, maxbibnames=2]{biblatex}

%Setup hyperref package, and colours for links
\usepackage[unicode, draft=false]{hyperref}
\definecolor{linkcolour}{rgb}{0,0.2,0.6}
\hypersetup{colorlinks,breaklinks,urlcolor=linkcolour,linkcolor=linkcolour}
\addbibresource{citations.bib}
\setlength\bibitemsep{1em}

%for social icons
\usepackage{fontawesome5}

%----------------------------------------------------------------------------------------
%	BEGIN DOCUMENT
%----------------------------------------------------------------------------------------
\begin{document}

% non-numbered pages
\pagestyle{empty} 

%----------------------------------------------------------------------------------------
%	TITLE
%----------------------------------------------------------------------------------------
\begin{tabularx}{\linewidth}{@{} C @{}}
\Huge{Nicolas Mendes Fernandes} \\[7.5pt]
\href{https://github.com/nicolas-mendes}{\raisebox{-0.05\height}\faGithub\ nicolas-mendes} \ $|$ \ 
\href{https://linkedin.com/in/nicolasmendesfernandes}{\raisebox{-0.05\height}\faLinkedin\ nicolasmendesfernandes} \ $|$ \ 
\href{mailto:nicolasmendes512@gmail.com}{\raisebox{-0.05\height}\faEnvelope \ nicolasmendes512@gmail.com} \\
\href{tel:+5541998710753}{\raisebox{-0.05\height}\faMobile \ +55 (41) 99871-0753} \\
\end{tabularx}

%Interests/ Keywords/ Summary
\section*{Sumário}
\centerline{\small\textit{Curitiba, PR}}
\vspace{4pt}

Estudante de Análise e Desenvolvimento de Sistemas com 6 meses de estudo dedicados ao desenvolvimento web, com foco prático na stack PHP, Laravel e SQL. Busco uma oportunidade de estágio para aplicar meu conhecimento na solução de problemas reais, absorver conhecimentos da equipe e aprofundar minhas habilidades nas ferramentas e práticas do ecossistema de desenvolvimento profissional.

%----------------------------------------------------------------------------------------
%	EDUCATION
%----------------------------------------------------------------------------------------
\section{Educação}
\begin{tabularx}{\linewidth}{@{}l X@{}}
    2025 - 2027 & 
    \begin{tabular}[t]{@{}l@{}} 
        Tecnólogo em Análise e Desenvolvimento de Sistemas \\
        \textbf{Universidade Federal do Paraná}
    \end{tabular}
    \hfill \normalsize (IRA: 0.877/1) \\
    
    \multicolumn{2}{@{}X@{}}{
        \addlinespace[2pt]
        \footnotesize{\textit{Tópicos: Desenvolvimento WEB, Administração de Sistemas, Prática de Programação, Engenharia de Requisitos, Bancos de Dados, Análise e Projeto de Sistemas}}
    } \\

    \addlinespace[5pt]

    2022 - 2024 & 
    \begin{tabular}[t]{@{}l@{}}
        Ensino Médio \\
        \textbf{Colégio Militar de Curitiba}
    \end{tabular}
    \hfill (Média Global: 9.02/10.0) \\
\end{tabularx}

%----------------------------------------------------------------------------------------
%	SKILLS
%----------------------------------------------------------------------------------------
\section{Habilidades Técnicas}
\begin{itemize}[noitemsep, nolistsep, leftmargin=*]
  \item Linguagens de Programação: PHP, Laravel, SQL
  \item Tecnologias WEB: HTML, CSS, BootStrap
  \item Banco de Dados: PostgreSQL, MySQL
  \item Sistemas Operacionais: Windows (CMD), Linux (Ubuntu) com prática em Shell Script (Bash).
  \item Controle de Versionamento: GIT, GitHub, GitLab
\end{itemize}

%----------------------------------------------------------------------------------------
%	PROJECTS
%----------------------------------------------------------------------------------------
\section*{Projetos}
\begin{itemize}[noitemsep, nolistsep, leftmargin=*, itemsep=8pt]
    \item 
        \textbf{Sistema CRUD com Foco em Segurança} \\
        Desenvolvi uma aplicação web completa de gerenciamento (CRUD), garantindo a proteção de dados do usuário contra as principais vulnerabilidades da web, através da arquitetura de código em PHP Orientado a Objetos (POO), implementação de prepared statements para prevenir SQL Injection, sanitização de outputs contra XSS e uso de hashing de senhas com `password_hash`.

    \item 
        \textbf{Conversor Histórico de Moeda} \\
        Criei uma ferramenta web interativa capaz de calcular e exibir o poder de compra histórico de valores em Reais com precisão, desenvolvendo a lógica em PHP para processar requisições a uma API REST externa, manipular os dados recebidos (JSON) e apresentar os resultados em uma interface responsiva com HTML/CSS e Bootstrap.

    \item 
        \textbf{Script de Backup Automatizado} \\
        Automatizei o processo de backup de diretórios locais, assegurando a integridade e a consistência dos dados com um sistema de verificação, através da criação de um script em Shell (Bash) com uma interface de menu interativa para o usuário, utilizando `rsync` para a sincronização eficiente e segura dos arquivos.
        
    \item 
        \textbf{Análise e Prototipação de Sistema de Gestão} \\
        Elaborei um plano de desenvolvimento de software para um sistema de gestão de eventos, resultando em um documento de requisitos claro e uma prototipação de alta fidelidade que validam as necessidades do negócio, aplicando técnicas de Engenharia de Requisitos para entrevistar stakeholders, modelar processos e definir o escopo do projeto.
\end{itemize}

%----------------------------------------------------------------------------------------
%  Languages
%----------------------------------------------------------------------------------------
\section{Idiomas}
\begin{tabularx}{\linewidth}{@{}X@{}}
\textbf{Inglês} — Intermediário (Leitura e Escrita) \\
\textbf{Português} — Nativo \\
\end{tabularx}

\vfill
\center{\footnotesize Last updated: \today}

\end{document}
