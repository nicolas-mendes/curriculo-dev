%-----------------------------------------------------------------------------------------------------------------------------------------------%
%	The MIT License (MIT)
%
%	Copyright (c) 2021 Jitin Nair
%
%	Permission is hereby granted, free of charge, to any person obtaining a copy
%	of this software and associated documentation files (the "Software"), to deal
%	in the Software without restriction, including without limitation the rights
%	to use, copy, modify, merge, publish, distribute, sublicense, and/or sell
%	copies of the Software, and to permit persons to whom the Software is
%	furnished to do so, subject to the following conditions:
%	
%	THE SOFTWARE IS PROVIDED "AS IS", WITHOUT WARRANTY OF ANY KIND, EXPRESS OR
%	IMPLIED, INCLUDING BUT NOT LIMITED TO THE WARRANTIES OF MERCHANTABILITY,
%	FITNESS FOR A PARTICULAR PURPOSE AND NONINFRINGEMENT. IN NO EVENT SHALL THE
%	AUTHORS OR COPYRIGHT HOLDERS BE LIABLE FOR ANY CLAIM, DAMAGES OR OTHER
%	LIABILITY, WHETHER IN AN ACTION OF CONTRACT, TORT OR OTHERWISE, ARISING FROM,
%	OUT OF OR IN CONNECTION WITH THE SOFTWARE OR THE USE OR OTHER DEALINGS IN
%	THE SOFTWARE.
%	
%
%-----------------------------------------------------------------------------------------------------------------------------------------------%

%----------------------------------------------------------------------------------------
%	DOCUMENT DEFINITION
%----------------------------------------------------------------------------------------

% article class because we want to fully customize the page and not use a cv template
\documentclass[a4paper,12pt]{article}

%----------------------------------------------------------------------------------------
%	FONT
%----------------------------------------------------------------------------------------

% % fontspec allows you to use TTF/OTF fonts directly
% \usepackage{fontspec}
% \defaultfontfeatures{Ligatures=TeX}

% % modified for ShareLaTeX use
% \setmainfont[
% SmallCapsFont = Fontin-SmallCaps.otf,
% BoldFont = Fontin-Bold.otf,
% ItalicFont = Fontin-Italic.otf
% ]
% {Fontin.otf}


%----------------------------------------------------------------------------------------
%	PACKAGES
%----------------------------------------------------------------------------------------
\usepackage[utf8]{inputenc}
\usepackage[T1]{fontenc}
\usepackage{lmodern}
\usepackage{url}
\usepackage{parskip} 	

%other packages for formatting
\RequirePackage{color}
\RequirePackage{graphicx}
\usepackage[usenames,dvipsnames]{xcolor}
\usepackage[scale=0.9]{geometry}

%tabularx environment
\usepackage{tabularx}

%for lists within experience section
\usepackage{enumitem}

% centered version of 'X' col. type
\newcolumntype{C}{>{\centering\arraybackslash}X} 

%to prevent spillover of tabular into next pages
\usepackage{supertabular}
\usepackage{tabularx}
\newlength{\fullcollw}
\setlength{\fullcollw}{0.47\textwidth}

%custom \section
\usepackage{titlesec}				
\usepackage{multicol}
\usepackage{multirow}
\usepackage{booktabs}

%CV Sections inspired by: 
%http://stefano.italians.nl/archives/26
\titleformat{\section}{\Large\scshape\raggedright}{}{0em}{}[\titlerule]
\titlespacing{\section}{0pt}{10pt}{10pt}

%for publications
\usepackage[style=authoryear,sorting=ynt, maxbibnames=2]{biblatex}

%Setup hyperref package, and colours for links
\usepackage[unicode, draft=false]{hyperref}
\definecolor{linkcolour}{rgb}{0,0.2,0.6}
\hypersetup{colorlinks,breaklinks,urlcolor=linkcolour,linkcolor=linkcolour}
\addbibresource{citations.bib}
\setlength\bibitemsep{1em}

%for social icons
\usepackage{fontawesome5}

%debug page outer frames
%\usepackage{showframe}


%----------------------------------------------------------------------------------------
%	BEGIN DOCUMENT
%----------------------------------------------------------------------------------------
\begin{document}

% non-numbered pages
\pagestyle{empty} 

%----------------------------------------------------------------------------------------
%	TITLE
%----------------------------------------------------------------------------------------

\begin{tabularx}{\linewidth}{@{} C @{}}
\Huge{Nicolas Mendes Fernandes} \\ [7.5pt]
\href{https://github.com/nicolas-mendes}{\raisebox{-0.05\height}\faGithub\ nicolas-mendes} \ $|$ \ 
\href{https://linkedin.com/in/nicolasmendesfernandes}{\raisebox{-0.05\height}\faLinkedin\ nicolasmendesfernandes} \\

\end{tabularx}

%----------------------------------------------------------------------------------------
%	CONTACT
%----------------------------------------------------------------------------------------

\section*{Contato}
\begin{center}
    \faMapMarker*{} Curitiba, PR \ $|$ \ 
    \faPhone\ \href{tel:+55419987107538}{(41) 999871-0753} \ $|$ \ 
    \faEnvelope\ \href{mailto:nicolasmendes512@gmail.com}{nicolasmendes512@gmail.com}
\end{center}



%----------------------------------------------------------------------------------------
%	SUMMARY
%----------------------------------------------------------------------------------------
\section*{Sumário}
Estudante de  \textbf{Análise e Desenvolvimento de Sistemas} com 6 meses de estudo dedicados ao \textbf{Desenvolvimento Web e Backend}, com foco prático na stack PHP, Laravel e SQL. \\
Busco uma oportunidade de estágio para aplicar meus estudos na solução de problemas reais, 
absorver conhecimentos da equipe e desenvolver minhas habilidades nas ferramentas e práticas do ecossistema de desenvolvimento profissional.

%----------------------------------------------------------------------------------------
%	EDUCATION
%----------------------------------------------------------------------------------------

\section{Educação}
\begin{tabularx}{\linewidth}{@{}l X@{}}
    2025 - 2027 & 
    \begin{tabular}[t]{@{}l@{}} 
        Tecnólogo em Análise e Desenvolvimento de Sistemas \\
        \textbf{Universidade Federal do Paraná}
    \end{tabular}
    \hfill \normalsize (IRA: 0.877/1) \\
    \multicolumn{2}{@{}X@{}}{
        \footnotesize{\textit{Tópicos: Desenvolvimento WEB, Administração de Sistemas, Prática de Programação, Engenharia de \mbox{Requisitos}, 
Bancos de Dados, Análise e Projeto de Sistemas}}
    } \\

    \addlinespace[8pt]

    2022 - 2024 & 
    \begin{tabular}[t]{@{}l@{}}
        Ensino Médio - \textbf{Colégio Militar de Curitiba}
    \end{tabular}
    \hfill (Média Global: 9.02/10.0) \\
\end{tabularx}


%----------------------------------------------------------------------------------------
%	SKILLS
%----------------------------------------------------------------------------------------

\section{Habilidades Técnicas}
\begin{itemize}[noitemsep, nolistsep, leftmargin=*]
  \item \textbf{Linguagens de Programação:} PHP, Laravel, SQL
  \item \textbf{Tecnologias WEB:} HTML, CSS, BootStrap
  \item \textbf{Banco de Dados:} PostgreSQL, MySQL
  \item \textbf{Sistemas Operacionais:} Windows (CMD), Linux (Ubuntu) com prática em Shell Script (Bash).
  \item \textbf{Controle de Versionamento:} GIT, GitHub, GitLab
\end{itemize}

%----------------------------------------------------------------------------------------
%	PROJECTS
%----------------------------------------------------------------------------------------

\section*{Projetos}
\begin{itemize}[noitemsep, nolistsep, leftmargin=*, itemsep=8pt]
    \item 
        \href{https://github.com/nicolas-mendes/CRUD_PHP}{\textbf{Sistema CRUD com Foco em Segurança}} \\ 
Desenvolvi uma aplicação web completa de \textbf{CRUD} (Create, Read, Update, Delete), projetada para o gerenciamento visual e intuitivo de um banco de dados. \\
Utilizando \textbf{PHP Orientado a Objetos}, \textbf{MySQL}, e \textbf{Bootstrap (HTML/CSS)}, o sistema possui um \textbf{forte foco em segurança}. As proteções incluem prepared statements contra \textbf{SQL Injection}, sanitização de saídas para prevenir ataques \textbf{XSS}, e \textbf{criptografia de senhas} com hashing (password\_hash). A aplicação também conta com uma funcionalidade de filtragem de registros e utiliza \textbf{UUIDv4}  para identificadores únicos e seguros.

    \item 
        \href{https://github.com/nicolas-mendes/consulta-historica-salario-minimo}{\textbf{Calculadora Histórica de Salário Mínimo}} \\
        Desenvolvi uma aplicação web para calcular a equivalência histórica de um salário em salários mínimos. Implementei a lógica em PHP para consumir a \textbf{API REST} do Banco Central do Brasil (SGS), tratando a resposta JSON para realizar os cálculos e exibir o resultado de forma dinâmica na interface.

    \item 
        \href{https://github.com/nicolas-mendes/script-backup-linux}{\textbf{Script de Backup Automatizado Linux}} \\
        Automatizei o processo de backup de diretórios locais, assegurando a integridade e a consistência dos dados com um sistema de verificação, através da criação de um script em Shell (Bash) com uma interface de menu interativa para o usuário, utilizando a função \textbf{`rsync`} para a sincronização eficiente e segura dos arquivos.
        
\end{itemize}

%----------------------------------------------------------------------------------------
%  LANGUAGES
%----------------------------------------------------------------------------------------
\section{Idiomas}
\begin{tabularx}{\linewidth}{@{}X@{}}
\textbf{Inglês} — Intermediário (Leitura, Escrita e Conversação) \\
\textbf{Português} — Nativo \\
\end{tabularx}

%----------------------------------------------------------------------------------------
%  FOOTER
%----------------------------------------------------------------------------------------
\vfill
\center{\footnotesize Última Atualização: \today}
\end{document}
